\doifnot{\contextmark}{LMTX}{
    \errhelp{LMTX/MkXL is required to compile this file.}
    \errmessage{Fatal error, exiting.}
}

\environment lwc-manual

\let\ac=\acronym
\let\q=\quotation

\setuplayout[width=43em]

\begingroup
    \tt
    \ctxlua{tt_font = font.current()}
\endgroup

\definecolor[code][x=e6ebe6]

\definetextbackground[code][
    location=always,
    background=color,
    backgroundcolor=code,
    topoffset=1ex,
    leftoffset=1ex,
    rightoffset=1ex,
    bottomoffset=1ex,
    frame=on,
    framecolor=code,
    rulethickness=1pt,
    corner=round,
    radius=4ex,
]

\def\startcode[#1]{%
    \def\codelanguage{#1}%
    \starttextbackground[code]%
    \interlinepenalty=10000%
    \vskip-1ex%
    \noindent{\ssx\sc example}\par%
    \grabbufferdata[code][startcode][stopcode]%
}

\startluacode
    local highlight <const> = require("syntax")

    interfaces.implement {
        name = "lua",
        public = true,
        arguments = { "string" },
        actions = function(code)
            highlight("lua", code, tt_font)
        end
    }

    interfaces.implement {
        name = "stopcode",
        public = true,
        actions = function()
            local language = tokens.getters.macro("codelanguage")
            local code = buffers.raw("code")
            highlight(language, code, tt_font, tex.leftskip.width)
            context.stoptextbackground()
            context.blank { "halfline" }
        end
    }

    local function is_tex(line)
        line = line or ""
        return line:match("^%-%-= (.*) %-%-%-$") or
               line:match("^%-%-= (.*)$") or
              (line:match("^[-=]+$") and "")
    end

    local function trim_dash(str)
        if str and str:match("^[%s-]*$") then
            return str:gsub("-", " ")
        else
            return str
        end
    end

    local function is_type(line)
        local cmd, name, extra, desc = line:match(
            "^%-%-%- (@%S+%s+)(%S+)(%s+%S+%s+)(.*)$"
        )
        if not cmd then
            desc = line:match("^%-%-%-%s%s+(.*)$")
        end

        return cmd, trim_dash(name), trim_dash(extra), trim_dash(desc)
    end

    local function is_blank(line)
        return (line or ""):match("^%s*$")
    end

    interfaces.implement {
        name = "processfile",
        public = true,
        arguments = { "string" },
        actions = function(filename)
            local lines <const> = io.loaddata(filename):splitlines()

            context.stepwise(function() print(xpcall(function()
                local text
                for index, line in ipairs(lines) do
                    local this_cmd, this_name, this_extra, this_desc = is_type(line)

                    local prev_tex = is_tex(lines[index - 1])
                    local this_tex = is_tex(line)
                    local next_tex = is_tex(lines[index + 1])

                    local prev_blank = is_blank(lines[index - 1])
                    local this_blank = is_blank(line)
                    local next_blank = is_blank(lines[index + 1])

                    if this_cmd then
                        context.par()
                        context.noindent()
                        context.verbatim.typ(this_cmd)
                        context.bold(context.nested.typ(this_name))
                        context.verbatim.typ(this_extra)
                    end

                    if this_desc then
                        context(this_desc .. " ")
                    end

                    if this_tex then
                        text = (text or [[\noindent ]]) .. this_tex .. "\n"
                    elseif text then
                        buffers.assign("content", text)
                        context.getbuffer { "content" }
                        context.blank { "quarterline" }
                        context.step()
                        text = nil
                    end

                    if not this_tex and not this_desc then
                        if not this_blank or
                           (this_blank and not (prev_blank or prev_tex))
                        then
                            context.par()
                            context.step()
                            highlight("lua", line, tt_font)
                        end
                    end
                end
            end, debug.traceback)) end)
        end
    }
\stopluacode

\setuphead[subsection][
    alternative=normal,
    style=\ssbfa,
    after={\blank[disable, penalty:10000]},
    continue=yes,
]

\catcode`\|=12

\startdocument[
    title=luatools,
    author=Max Chernoff,
    version=0.0.0, %%version
    github=https://github.com/gucci-on-fleek/luatools,
]
    \subject{Contents}
    \placecontent[criterium=all]

    \processfile{../source/luatools.lua}
\stopdocument
